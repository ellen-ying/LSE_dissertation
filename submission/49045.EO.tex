% Options for packages loaded elsewhere
\PassOptionsToPackage{unicode}{hyperref}
\PassOptionsToPackage{hyphens}{url}
%
\documentclass[
  11pt,
]{article}
\usepackage{amsmath,amssymb}
\usepackage{lmodern}
\usepackage{iftex}
\ifPDFTeX
  \usepackage[T1]{fontenc}
  \usepackage[utf8]{inputenc}
  \usepackage{textcomp} % provide euro and other symbols
\else % if luatex or xetex
  \usepackage{unicode-math}
  \defaultfontfeatures{Scale=MatchLowercase}
  \defaultfontfeatures[\rmfamily]{Ligatures=TeX,Scale=1}
\fi
% Use upquote if available, for straight quotes in verbatim environments
\IfFileExists{upquote.sty}{\usepackage{upquote}}{}
\IfFileExists{microtype.sty}{% use microtype if available
  \usepackage[]{microtype}
  \UseMicrotypeSet[protrusion]{basicmath} % disable protrusion for tt fonts
}{}
\makeatletter
\@ifundefined{KOMAClassName}{% if non-KOMA class
  \IfFileExists{parskip.sty}{%
    \usepackage{parskip}
  }{% else
    \setlength{\parindent}{0pt}
    \setlength{\parskip}{6pt plus 2pt minus 1pt}}
}{% if KOMA class
  \KOMAoptions{parskip=half}}
\makeatother
\usepackage{xcolor}
\usepackage[margin=1.0in]{geometry}
\usepackage{graphicx}
\makeatletter
\def\maxwidth{\ifdim\Gin@nat@width>\linewidth\linewidth\else\Gin@nat@width\fi}
\def\maxheight{\ifdim\Gin@nat@height>\textheight\textheight\else\Gin@nat@height\fi}
\makeatother
% Scale images if necessary, so that they will not overflow the page
% margins by default, and it is still possible to overwrite the defaults
% using explicit options in \includegraphics[width, height, ...]{}
\setkeys{Gin}{width=\maxwidth,height=\maxheight,keepaspectratio}
% Set default figure placement to htbp
\makeatletter
\def\fps@figure{htbp}
\makeatother
\setlength{\emergencystretch}{3em} % prevent overfull lines
\providecommand{\tightlist}{%
  \setlength{\itemsep}{0pt}\setlength{\parskip}{0pt}}
\setcounter{secnumdepth}{5}
\newcommand{\bcenter}{\begin{center}}
\newcommand{\ecenter}{\end{center}}
\newcommand{\btitlepage}{\begin{titlepage}}
\newcommand{\etitlepage}{\end{titlepage}}
\usepackage{helvet}
\renewcommand*\familydefault{\sfdefault}
\usepackage{setspace}\onehalfspacing
\usepackage{booktabs}
\usepackage[font=small,labelfont=bf]{caption}
\ifLuaTeX
  \usepackage{selnolig}  % disable illegal ligatures
\fi
\IfFileExists{bookmark.sty}{\usepackage{bookmark}}{\usepackage{hyperref}}
\IfFileExists{xurl.sty}{\usepackage{xurl}}{} % add URL line breaks if available
\urlstyle{same} % disable monospaced font for URLs
\hypersetup{
  hidelinks,
  pdfcreator={LaTeX via pandoc}}

\author{}
\date{\vspace{-2.5em}}

\begin{document}

\begin{titlepage}

\begin{center}

\vspace*{30mm}

Candidate number: 49045

\vspace*{10mm}

\hypertarget{title-here}{%
\section*{TITLE HERE}\label{title-here}}
\addcontentsline{toc}{section}{TITLE HERE}

\vspace*{10mm}

Supervisor: Ekaterina (Katya) Oparina\\

Word count:

\vspace*{30mm}

Submitted as partial fulfilment for\\

MSc in Behavioural Science\\

Department of Psychological and Behavioural Science\\

The London School of Economics and Political Science

\end{center}

\end{titlepage}

\newpage

Total word limit: 10000

\hypertarget{abstract-100-150}{%
\subsection*{Abstract (100-150)}\label{abstract-100-150}}
\addcontentsline{toc}{subsection}{Abstract (100-150)}

\newpage

\hypertarget{introduction-1000}{%
\section{Introduction (1000)}\label{introduction-1000}}

Gender gap in labour force participation is a world-wide phenomenon
which is particularly pronounced in developing countries. Globally, the
rate of labour force participation is about 75\% among men while only
50\% among women \textbf{(International Labour Organization, 2021)}. In
regions such as Middle East, North Africa, and South Asia, the gender
gap is even greater, with around 75\% of men and 20\% of women
participating in the labour force \textbf{(International Labour
Organization, 2021)}. In the search of potential measures to narrow the
gender gap, its causes have been extensively studied with a recent
attention on the prominent role of social norms. Researchers speculate
that interventions aiming at changing social norms may be the key to
achieve greater gender equality in labour force participation
\textbf{(Bursztyn et al., 2020; Codazzi et al., 2018; Jayachandran,
2021)}.

A recent successful attempt in developing such social norm interventions
was made by Bursztyn and colleagues \textbf{(2020)} in Saudi Arabia. In
Saudi Arabia, gender norms exist that expect women to be absent from
labour or segregated from men at workplace, and that women need approval
from their male ``guardian'' (usually the husband or father) if they
want to work outside the home \textbf{(Bursztyn et al., 2018)}. These
allow us to make a reasonable speculation that the low female labour
participation rate \textbf{(28\% in 2022, International Labour
Organization)} may be because men don't allow their wives to work
outside the home (WWOH) due to the internalisation of the gender norms
(i.e., personal beliefs aligning with the norm that women should not
work outside the home). Interestingly, however, Bursztyn and colleagues
\textbf{(2020)} found that the critical factor was rather Saudi men's
misperception of the injunctive norms regarding WWOH. Their findings
revealed that 80-85\% of surveyed Saudi men who were married and aged
18-35 reported to agree with the statement that women should be allowed
to work outside the home, while more than three quarters of them
underestimated this percentage.

Based on this result, Bursztyn and colleagues \textbf{(2020)} designed
an intervention to correct men's misperception regarding WWOH found
evidence supporting evidence for its effectiveness in changing their
labour supply decisions. They found that men who received correct
information of the true percentage of supporting men (vs.~those who did
not receive the information) were significantly more likely to sign up
for job matching service for their wives immediately after the
intervention. Their wives were also more likely to have applied and
interviewed for a job three to five months after the intervention. The
researchers also tested the effect of a similar intervention on Saudi
women in a field setting, finding that women who received the
information on the percentage of men supporting WWOH (vs.~those who did
not received the information) were more likely to take up a part-time
job outside the home, the alternative of which is a job that can be done
at home.

The present study is inspired by this promising intervention method of
correcting the misperception of social norm.

\hypertarget{literature-review-2000}{%
\section{Literature Review (2000)}\label{literature-review-2000}}

\hypertarget{methodology-2000-3000}{%
\section{Methodology (2000-3000)}\label{methodology-2000-3000}}

\hypertarget{results-2000-2500}{%
\section{Results (2000-2500)}\label{results-2000-2500}}

\hypertarget{discussion-and-conclusion-1000-1500}{%
\section{Discussion and Conclusion
(1000-1500)}\label{discussion-and-conclusion-1000-1500}}

\hypertarget{reference}{%
\section{Reference}\label{reference}}

\hypertarget{appendices}{%
\section{Appendices}\label{appendices}}

\end{document}
