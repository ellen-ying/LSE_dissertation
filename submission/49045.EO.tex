% Options for packages loaded elsewhere
\PassOptionsToPackage{unicode}{hyperref}
\PassOptionsToPackage{hyphens}{url}
%
\documentclass[
  11pt,
]{article}
\usepackage{amsmath,amssymb}
\usepackage{lmodern}
\usepackage{iftex}
\ifPDFTeX
  \usepackage[T1]{fontenc}
  \usepackage[utf8]{inputenc}
  \usepackage{textcomp} % provide euro and other symbols
\else % if luatex or xetex
  \usepackage{unicode-math}
  \defaultfontfeatures{Scale=MatchLowercase}
  \defaultfontfeatures[\rmfamily]{Ligatures=TeX,Scale=1}
\fi
% Use upquote if available, for straight quotes in verbatim environments
\IfFileExists{upquote.sty}{\usepackage{upquote}}{}
\IfFileExists{microtype.sty}{% use microtype if available
  \usepackage[]{microtype}
  \UseMicrotypeSet[protrusion]{basicmath} % disable protrusion for tt fonts
}{}
\makeatletter
\@ifundefined{KOMAClassName}{% if non-KOMA class
  \IfFileExists{parskip.sty}{%
    \usepackage{parskip}
  }{% else
    \setlength{\parindent}{0pt}
    \setlength{\parskip}{6pt plus 2pt minus 1pt}}
}{% if KOMA class
  \KOMAoptions{parskip=half}}
\makeatother
\usepackage{xcolor}
\usepackage[margin=1.0in]{geometry}
\usepackage{graphicx}
\makeatletter
\def\maxwidth{\ifdim\Gin@nat@width>\linewidth\linewidth\else\Gin@nat@width\fi}
\def\maxheight{\ifdim\Gin@nat@height>\textheight\textheight\else\Gin@nat@height\fi}
\makeatother
% Scale images if necessary, so that they will not overflow the page
% margins by default, and it is still possible to overwrite the defaults
% using explicit options in \includegraphics[width, height, ...]{}
\setkeys{Gin}{width=\maxwidth,height=\maxheight,keepaspectratio}
% Set default figure placement to htbp
\makeatletter
\def\fps@figure{htbp}
\makeatother
\setlength{\emergencystretch}{3em} % prevent overfull lines
\providecommand{\tightlist}{%
  \setlength{\itemsep}{0pt}\setlength{\parskip}{0pt}}
\setcounter{secnumdepth}{5}
\newcommand{\bcenter}{\begin{center}}
\newcommand{\ecenter}{\end{center}}
\newcommand{\btitlepage}{\begin{titlepage}}
\newcommand{\etitlepage}{\end{titlepage}}
\usepackage{helvet}
\renewcommand*\familydefault{\sfdefault}
\usepackage{setspace}\onehalfspacing
\usepackage{booktabs}
\usepackage[font=small,labelfont=bf]{caption}
\ifLuaTeX
  \usepackage{selnolig}  % disable illegal ligatures
\fi
\IfFileExists{bookmark.sty}{\usepackage{bookmark}}{\usepackage{hyperref}}
\IfFileExists{xurl.sty}{\usepackage{xurl}}{} % add URL line breaks if available
\urlstyle{same} % disable monospaced font for URLs
\hypersetup{
  hidelinks,
  pdfcreator={LaTeX via pandoc}}

\author{}
\date{\vspace{-2.5em}}

\begin{document}

\begin{titlepage}

\begin{center}

\vspace*{30mm}

Candidate number: 49045

\vspace*{10mm}

\hypertarget{title-here}{%
\section*{TITLE HERE}\label{title-here}}
\addcontentsline{toc}{section}{TITLE HERE}

\vspace*{10mm}

Supervisor: Ekaterina (Katya) Oparina\\

Word count:

\vspace*{30mm}

Submitted as partial fulfilment for\\

MSc in Behavioural Science\\

Department of Psychological and Behavioural Science\\

The London School of Economics and Political Science

\end{center}

\end{titlepage}

\newpage

Total word limit: 10000

\hypertarget{abstract-100-150}{%
\section*{Abstract (100-150)}\label{abstract-100-150}}
\addcontentsline{toc}{section}{Abstract (100-150)}

\newpage

\hypertarget{introduction-1000}{%
\section{Introduction (1000)}\label{introduction-1000}}

Gender gap in labour force participation is a world-wide phenomenon
which is particularly pronounced in developing countries. Globally, the
rate of labour force participation is about 75\% among men while only
50\% among women \textbf{(International Labour Organization, 2021)}. In
regions such as Middle East, North Africa, and South Asia, the gender
gap is even greater, with around 75\% of men and 20\% of women
participating in the labour force \textbf{(International Labour
Organization, 2021)}. In the search of potential measures to narrow the
gender gap, its causes have been extensively studied with a recent
attention on the prominent role of social norms. Researchers speculate
that interventions aiming at changing social norms may be the key to
achieve greater gender equality in labour force participation
\textbf{(Bursztyn et al., 2020; Codazzi et al., 2018; Jayachandran,
2021)}.

A recent successful attempt in developing such social norm interventions
was made by Bursztyn and colleagues \textbf{(2020)} in Saudi Arabia. In
Saudi Arabia, gender norms exist that expect women to be absent from
labour or segregated from men at workplace, and that women need approval
from their male ``guardian'' (usually the husband or father) if they
want to work outside the home \textbf{(Bursztyn et al., 2018)}. These
allow us to make a reasonable speculation that the low female labour
participation rate \textbf{(28\% in 2022, International Labour
Organization)} may be because men don't allow their wives to work
outside the home (WWOH) due to the internalisation of the gender norms
(i.e., personal beliefs aligning with the norm that women should not
work outside the home). Interestingly, however, Bursztyn and colleagues
\textbf{(2020)} found that the critical factor was rather Saudi men's
misperception of the injunctive norms regarding WWOH. Their findings
revealed that 80-85\% of surveyed Saudi men who were married and aged
18-35 reported to agree with the statement that women should be allowed
to work outside the home, while more than three quarters of them
underestimated this percentage.

Based on this result, Bursztyn and colleagues \textbf{(2020)} designed
an intervention to correct men's misperception regarding WWOH found
evidence supporting evidence for its effectiveness in changing their
labour supply decisions. They found that men who received correct
information of the true percentage of supporting men (vs.~those who did
not receive the information) were significantly more likely to sign up
for job matching service for their wives immediately after the
intervention. Their wives were also more likely to have applied and
interviewed for a job three to five months after the intervention. The
researchers also tested the effect of a similar intervention on Saudi
women in a field setting, finding that women who received the
information on the percentage of men supporting WWOH (vs.~those who did
not received the information) were more likely to take up a part-time
job outside the home instead of a position to work at home.

The present research takes correcting norm misperception as a promising
type of social norm intervention and seeks to explore its effectiveness
when scaled up. Since large-scale policy intervention can usually be
costly to implement and test, the present research aims to study, prior
to implementation, the intervention strategy that is theoretically the
most effective under appropriate assumptions via agent-based modelling
(ABM). The reason for choosing ABM as the method of modelling the system
of WWOH action is that the dynamic interaction among people's private
belief and norm perception regarding WWOH, as well as their labour
supply decision and action can be viewed as complex \textbf{(Johnson,
2009)}. One's labour supply decision and action can presumably influence
their acquaintances' private belief and norm perception, and these
influences in turn impact their acquaintances' labour supply decision
and action as a function of private belief and norm perception,
eventually creating a feedback loop among individuals. This nature of
having agents interacting dynamically and irregularly over time makes
the system of WWOH action irreducible to lower-level descriptions (e.g.,
mathematical equations) and qualifies it as a complex system
\textbf{(Gustafsson \& Sternad, 2010)}. ABMs are suitable for studying
the dynamics and emergent properties of such systems, as well as
simulate and explore the consequences of policy interventions to them
\textbf{(Bailey et al., 2019; Madsen et al., 2019)}.

\textbf{give space for specific research questions}

\hypertarget{literature-review-2000}{%
\section{Literature Review (2000)}\label{literature-review-2000}}

\hypertarget{models-of-belief-and-action-dynamics}{%
\subsection{Models of Belief and Action
Dynamics}\label{models-of-belief-and-action-dynamics}}

Belief and action changes among a group of interconnected and mutually
influencing agents have been studied widely through mathematical and
computational models. These models assume that agents hold beliefs
regarding certain issues. At each time step, these beliefs are modified
based on their neighbours' beliefs according to some rule. The dynamics
of belief distribution in a population over time is studied. This
section will first review various methods to formally model agents'
beliefs and actions and the ways in which agents influence one another.
Then, the topics that these models are usually applied to studied are
introduced.

\hypertarget{formal-representation-of-beliefs-and-actions}{%
\subsubsection{Formal Representation of Beliefs and
Actions}\label{formal-representation-of-beliefs-and-actions}}

Different models of belief and action dynamics model agents' beliefs and
actions and the ways in which they are affected by those of other agents
differently \textbf{(Hassani et al., 2022)}. The most simplistic models
represent agents' beliefs using binary variables, the classic ones of
which include the Ising model \textbf{(Li et al., 2019)}, the voter
model \textbf{(Holley \& Liggett, 1975)}, and the Sznajd model
\textbf{(Sznajd-Weron \& Sznajd, 2000)}. Although these models and their
variations adopt different rules to update agents' beliefs, the updating
usually results in the agents being memoryless about their previous
beliefs. For example, in the voter model, an agent \(i\) is randomly
chosen at each time step, together with one of its neighbour \(j\), and
the agent \(i\) then abandons its previous belief takes the opinion of
its neighbour \(j\).

Other simple models represent agents' beliefs as continuous variables
that take the value of a real number. These are exemplified by the
classic Degroot model \textbf{(Berger, 1981)} and the bounded confidence
model \textbf{(Rainer \& Krause, 2002)}. The models of this type usually
updates agents' beliefs using a weighted average of each individual's
belief and those of their neighbour(s). For example, in the bounded
confidence model, an agent \(i\) is randomly chosen at each time step
\(t\), together with one of its neighbour \(j\), who hold the beliefs
\(\sigma_i(t)\) and \(\sigma_j(t)\), respectively. When the condition
\(|\sigma_i(t) - \sigma_j(t)| < \epsilon\) is met, the agent \(i\)
updates its belief according to
\(\sigma_i(t+1) = (1 - \alpha)\sigma_i(t) + \alpha\sigma_j(t)\), and the
agent \(j\) according to
\(\sigma_j(t+1) = (1 - \alpha)\sigma_j(t) + \alpha\sigma_i(t)\). In
other words, when the selected agents hold similar enough beliefs
according to a threshold \(\epsilon\), they update their beliefs as the
weighted average of their neighbour's and their own previous beliefs
based on a convergence parameter \(\alpha\).

More sophisticated models recognise the distinction between agents'
private beliefs and public actions, modelling the former as a continuous
variable and the latter a discrete one. The Continuous Opinions and
Discrete Actions (CODA) model \textbf{(Martins, 2013)} and the more
recent social network opinions and actions evolutions (SNOAEs) model
\textbf{(Zhan et al., 2022)} are examples of this category. To take the
CODA model as an example, it models private beliefs as
\(P(A) \in [0,1]\), the probability of the \(A\) being the best
alternative, which is also the probability of an agent publicly
displaying the action \(A\). Agents have no access to other agents'
private beliefs but only their public actions, which is used to update
one's own private beliefs via the Bayes' theorem. For example, upon
observing a neighbour \(j\) displaying the action \(A\) at the time step
\(t\), an agent update its private beliefs according to the rule
\(P_{t+1}(A) = P(A | a_j = +1) \propto P_t(A) P(a_j = +1 | A)\), where
\(P(A | a_j = +1)\) denotes the probability of \(A\) being the best
action conditioned on the neighbour \(j\) displaying \(A\), and
\(P(a_j = +1 | A)\) the probability of the neighbour \(j\) displaying
\(A\) conditioned on \(A\) being the best action (which is modelled as a
constant).

It is important to notice that there is a recent trend to build
psychologically realistic models to represent belief and behaviour
dynamics among a group of agents \textbf{(Duggins, 2017; Gavrilets,
2021; Tverskoi et al., 2023)}. Still assuming each agent's beliefs and
actions are affected by their neighbours, these model further consider
the influence of social norm perception, the psychological tendency to
conform to peers, external authorities, as well as material cost-benefit
considerations \textbf{(Gavrilets, 2021)}. To incorporate these
psychological and social complexities, a recent unifying modelling
framework \textbf{(Gavrilets, 2021)} not only considers the distinction
between agents' private beliefs and public actions, but also model
agents' perception of others' private beliefs and public actions as two
separate variables. This allows the representation of various
psychological factors in updating beliefs and actions, including
cognitive dissonance in taking a private belief (i.e., the aversion
towards misaligned belief and action), social projection in perceiving
others' private beliefs (i.e., the tendency to project one's own private
belief onto others), and compliance with authority.

\hypertarget{model-applications}{%
\subsubsection{Model Applications}\label{model-applications}}

Models of belief and action dynamics have been applied to studying
multiple themes associated with opinion change in a population. These
include but are not limited to consensus reaching and polarisation
\textbf{(Acemoglu \& Ozdaglar, 2010; Hassani et al., 2022; Jager \&
Amblard, 2004; Duggins, 2017)}, the spread of (mis)information
\textbf{(Watts, 2002; Watts \& Dodds, 2007; Pilditch et al., 2022)}, and
echo chamber formation \textbf{(Madsen et al., 2018; Fränken \&
Pilditch, 2021)}. Applying mathematical or computational models,
researchers are mostly interested in the assumptions and conditions
under which certain phenomena can arise.

For instance, a study reviews a series of mathematical models to answer
the question about consensus reaching, information aggregation, and the
spread of misinformation. \textbf{\emph{More specifically, it answers
the question about under what conditions 1) agents can hold beliefs in
an agreement even when they start with different views, 2) information
can be aggregated in the population so that agents holding incorrect
beliefs can end up with correct beliefs, and 3) prominent agents can
spread misinformation. The study considers both Bayesian (i.e., agents
update their beliefs via the Bayes' theorem) and non-Bayesian models
(i.e., agents update their beliefs via other rules) (consider remove)}}
and concludes that in both types of models there is a tendency towards
reaching consensus, but agents' beliefs are not effectively corrected
through information aggregation. Misinformation spreads with limited
extent in both types of models, which is due to the limited influence of
misinformation on Bayesian agents in Bayesian models, and the lack of
persistent disagreements in the population in non-Bayesian models.

In addition to belief updating mechanisms, other research has also
studied how network structures can influence the emergence of phenomena
related to opinion change. \textbf{\emph{Focusing on information
cascade, a phenomenon where a group of individuals make the same
decision sequentially,}} a simulation study finds that the distribution
of the size of cascades depends on the connectivity of the network, so
does the type of agents who can easily trigger a cascade \textbf{(Watts,
2002)}. A subsequent simulation study further points out that
influential individuals, i.e., those who have the highest number of
connections to others, are not sufficient for triggering cascades, but
the existence of a group of other individuals who are easily influenced
by the influential ones is critical \textbf{(Watts \& Dodds, 2007)}.

\hypertarget{agent-based-models-of-pluralistic-ignorance}{%
\subsection{Agent-Based Models of Pluralistic
Ignorance}\label{agent-based-models-of-pluralistic-ignorance}}

There has been a growing interest in using ABMs to study the conditions
under which the phenomenon called pluralistic ignorance (PI) can arise.
This line of research usually defines PI as the situation where the
majority of people in a population express opinions different from their
private beliefs \textbf{(e.g., Seeme et al., 2016; Wang et al., 2014; Ye
et al., 2019)}. The earliest work primarily focuses on the effects of
the properties of the network on holding inconsistent private beliefs
and expressed opinions \textbf{(Centola et al., 2005)}. The research
inserts, as an initial condition, a few agents who hold a private belief
held by few individuals in the population (referred to as ``true
believers'') and enforce other agents to adopt the same belief. It
reveals that holding inconsistent private beliefs and expressed opinions
(which is equivalent to enforcing the belief on other agents in the
context of this research) cannot spread widely in the population if the
population is fully connected, if the true believers are scattered in
the population rather than clustered, or if ties in the network are
randomly rewired, breaking the ties that originally exist between local
neighbours.

Recent studies start to include in their models psychological processes
that have been theorised by research in social psychology. Two processes
that have been mostly attended to are social conformity and cognitive
dissonance \textbf{(e.g., Wang et al., 2014; Seeme, 2019)}. For
instance, a research models the change in agents' expressed opinions as
a result of the influence from the opinions in the groups formed among
the neighbours, representing people's psychological tendency to conform
to the group \textbf{(Wang et al., 2014)}. Following the cognitive
dissonance theory, the model asks the agents change their private
attitudes only when the group influence is moderate. When the group
influence is strong, the model represents the situation where people are
aware of the groups influence and doesn't ask agents to change their
private beliefs. Under these assumptions, the research finds that there
is a widespread inconsistency between agents' private beliefs and
expressed opinions, that is, pluralistic ignorance exists by definition.

Another example is a study that considers social conformity and
cognitive dissonance assuming the rationality of agents \textbf{(Seeme,
2019)}. It models the agents as having a utility function that is
proportional to the rewards from high conformity with its opinion group
(expressing similar opinions with the mean opinion of the group) and the
rewards from low cognitive dissonance (holding similar private beliefs
and expressed opinions). Agents update their private beliefs and
expressed opinions by maximising the value of the utility function. The
study finds that different degrees of pluralistic ignorance arise under
different conditions. When all agents make the update simultaneously,
the more opinion groups in the population, the more agents hold
inconsistent private beliefs and expressed opinions; when the updating
happens sequentially (i.e., agents update one by one), all agents end up
holding inconsistent private beliefs and expressed opinions.

\hypertarget{existing-intervention-strategy}{%
\subsection{Existing intervention
strategy}\label{existing-intervention-strategy}}

Discuss existing social norm intervention that corrects misperception

\hypertarget{current-research}{%
\subsection{Current Research}\label{current-research}}

what's new here

ABMs are a type of computational model that simulates in a synthetic
environment the actions and interactions of autonomous agents, whose
behaviours determine the evolution of the entire system \textbf{(Bandini
et al., 2009)}. ABMs have been deployed in research in a wide range of
fields including climate science \textbf{(Simmonds et al., 2019)},
ecology \textbf{(McLane et al., 2011)}, epidemiology \textbf{(de Mooij
et al., 2022)}, economics \textbf{(Grazzini \& Richiardi, 2015)},
finance \textbf{(Bonabeau, 2022; Chen et al., 2017,)}, and social
sciences \textbf{(Epstein \& Axtell, 1996; Schelling, 2006)}. Recently,
findings from psychological and cognitive science have also been
integrated in building more psychologically realistic agent-based models
to study political opinion dynamics \textbf{(Duaggins, 2017)}, the
formation of echo chamber \textbf{(Fränken \& Pilditch, 2021; Madsen et
al., 2018)}, and interventions to the spread of misinformation
\textbf{(Pilditch et al., 2022)}. These recent developments exemplify
the ability of ABMs in modelling a system where multiple
individual-level processes, such as interpersonal influence, social
conformity, and commitment to previous beliefs, together influence the
emergent properties of the system. Since the system of WWOH actions is
presumably such a system, ABM is able to model it (???).

\hypertarget{methodology-2000-3000}{%
\section{Methodology (2000-3000)}\label{methodology-2000-3000}}

\hypertarget{results-2000-2500}{%
\section{Results (2000-2500)}\label{results-2000-2500}}

\hypertarget{discussion-and-conclusion-1000-1500}{%
\section{Discussion and Conclusion
(1000-1500)}\label{discussion-and-conclusion-1000-1500}}

\hypertarget{reference}{%
\section{Reference}\label{reference}}

\hypertarget{appendices}{%
\section{Appendices}\label{appendices}}

\end{document}
